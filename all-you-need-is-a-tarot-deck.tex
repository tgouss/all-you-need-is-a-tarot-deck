\documentclass{book}
%\usepackage[utf8]{inputenc}
\usepackage{lmodern}
\usepackage{geometry}
\geometry{a4paper, margin=1.9cm}


\title{Jeu de cartes à tout faire !}
\author{theophile.gousselot }
\date{Février 2022}


\renewcommand*\contentsname{Les jeux !}
\usepackage{graphicx}
\usepackage{color}

%\usepackage{sectsty}
%\sectionfont{\fontsize{12}{15}\selectfont}

% Tikz / PGF
\usepackage{tikz}
\usetikzlibrary{matrix,arrows,calc,shapes.geometric,shapes.misc,shapes.symbols,shapes.arrows,automata,through,positioning,scopes,decorations.shapes,decorations.text,decorations.pathmorphing,shadows}
\usetikzlibrary{arrows,decorations.pathmorphing,backgrounds,fit,positioning,shapes.symbols,chains}
\usepackage{pgfplots, pgfplotstable}
\pgfplotsset{compat=1.7}

\usepackage{fancyhdr} 
\fancyhf{}
\cfoot{\thepage}
\pagestyle{fancy} 

\usepackage{minitoc}
\renewcommand{\mtifont}{\Large\sffamily\tt}
\renewcommand{\mtcfont}{\sffamily\tt}
\renewcommand{\mtcSfont}{\sffamily\tt}
\renewcommand{\mtcSSfont}{\sffamily\tt}
\renewcommand{\mtcSSSfont}{\sffamily\tt}

\dominitoc[n]
\mtcsetrules{*}{off}

\usepackage{type1cm}
\usepackage{titlesec, blindtext}
\definecolor{gray75}{gray}{0.75}
\newcommand{\hsp}{\hspace{20pt}}
\titleformat{\chapter}[display]
{\normalfont\fontsize{40pt}{42pt}\selectfont \tt}{\chaptertitlename\ \thechapter}{20pt}{\fontsize{48pt}{50pt}\selectfont }
\titleformat{\section}
{\normalfont\huge\tt}{\thesection}{1em}{}
\titleformat{\subsection}
{\normalfont\large\tt}{\thesubsection}{1em}{}
\titleformat{\subsubsection}
{\normalfont\normalsize\tt}{\thesubsubsection}{1em}{}
\titleformat{\paragraph}[runin]
{\normalfont\normalsize\bf\tt}{\theparagraph}{1em}{}
\titleformat{\subparagraph}[runin]
{\normalfont\normalsize\tt}{\thesubparagraph}{1em}{}




\usepackage{hyperref}
\hypersetup{
    colorlinks,
    citecolor=black,
    filecolor=black,
    linkcolor=black,
    urlcolor=black
}


\usepackage{ulem}


%\newcommand{\obj}[0]{\Large \fontseries{b}\selectfont Objectif}
%\newcommand{\ins}[0]{\Large \fontseries{b}\selectfont Installation}
%\newcommand{\der}[0]{\Large \fontseries{b}\selectfont Déroulement}
%\newcommand{\vic}[0]{\Large \fontseries{b}\selectfont Victoire}
%\newcommand{\rol}[0]{\Large \fontseries{b}\selectfont Rôles des cartes}

\newcommand{\obj}[0]{{ \vspace{0.5cm} \noindent \Large  \uline{Objectif :}}}
\newcommand{\ins}[0]{{ \vspace{0.5cm} \noindent \Large  \uline{Installation :}}}
\newcommand{\der}[0]{{ \vspace{0.5cm} \noindent \Large  \uline{Déroulement :}}}
\newcommand{\vic}[0]{{ \vspace{0.5cm} \noindent \Large  \uline{Victoire :}}}
\newcommand{\rol}[0]{{ \vspace{0.5cm} \noindent \Large  \uline{Rôles des cartes :}}}

\let\olditemize\itemize
\renewcommand\itemize{\olditemize\addtolength{\itemsep}{-0.2cm}}

\begin{document}
\tt
\begin{center}


\huge Jeu de cartes à tout faire ! \\


\Large Théophile Gousselot\\
2022
\end{center}


\tableofcontents

\newpage

\section*{Introduction}
\subsection*{Préambule}

\subsection*{Matériel requis}
\begin{itemize}
    \item Un jeu de carte de tarot et c'est tout !
\end{itemize}


 
\chapter{Les Jeux de Cartes}
\minitoc
\newpage

\section{8 américain (UNO)}


\obj
\begin{itemize}
  \item Se débarrasser de toutes ses cartes.
  \item[] {\Large \fontseries{b}\selectfont {OU}}  Minimiser son nombre de points.
\end{itemize}

\ins
\begin{itemize}
  \item Distribuer 5 cartes par personne.
  \item Déposer 1 carte face visible.
  \item Former la pioche avec le reste.
\end{itemize}


\der
\begin{itemize}
  \item À tour de rôle, chaque joueur doit :
    \begin{itemize}
      \item Recouvrir la carte visible d'une carte de la même couleur/valeur ou d'un joker.
      \item Piocher une carte et ne pas poser de carte.
    \end{itemize}
  \item Lorsqu'un joueur dépose son avant-dernière carte il doit annoncer ``UNO'', si un autre joueur annonce avant lui ``CONTRE-UNO'' alors il doit piocher deux cartes.
\end{itemize}

\vic
\begin{itemize}
  \item Le premier joueur s'étant débérassé de toutes ses cartes gagne la manche. Les autres joueurs marquent :
  \begin{itemize}
    \item 20 points par habillé et As.
    \item Autant de points que le numéro sur la carte.
  \end{itemize}
\end{itemize}


\rol
\begin{itemize}
    \item Cavalier : Change de couleur.
    \item As :  Le joueur suivant pioche 2 cartes.
    \item Valet : Change de sens.
    \item Joker : Annule une attaque (à jouer au tour des autres) .
    \item Roi : Le joueur rejoue.
    \item Reine : Le joueur suivant saute son tour.
\end{itemize}

\begin{tikzpicture}[remember picture, overlay, shift={(current page.north east)},  x={(current page.south east)}, y={(current page.north west)}]
  \node[below left] at (-1.4cm,-1.4cm){\includegraphics[width=8cm,height=8cm,keepaspectratio]{8americain-uno/8americain-uno.png}};
\end{tikzpicture}


\newpage
\chapter{Les Jeux Alternatifs}
\minitoc

\newpage
\section{Le Lama}
\section{Mafia de cuba}
\section{Stupid vautour}
\section{The Crew - La foule}
\section{Mafia de cuba}
\section{.}
\section{.}
\section{.}


\newpage
\section{TODO}
\begin{itemize}
    \item lister tous les jeux
    \item définir format (photo, objectif, préparation, déroulement, victoire, réprésentation/traduction des cartes)
\end{itemize}
\end{document}

